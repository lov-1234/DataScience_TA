\documentclass{article}
\usepackage{amsmath}
\usepackage{amsfonts}
\usepackage{amssymb}
\usepackage{hyperref}
\usepackage{graphicx}
\usepackage{enumitem}

\title{{\normalsize Tutorial 1:} \\ Introduction to Data Science \\~~\\ {\bf What is Data? - Solutions} }
\author{Lovnesh Bhardwaj}
\date{September 16, 2024}

\begin{document}
\maketitle

\section*{Exercises}

\begin{enumerate}

\item 
{\bf Identify Statistical Units}\\
The statistical units in the table provided are the students Alice and Bob. These units are described by the following features:
\begin{itemize}
	\item Name
	\item Age
	\item Major
\end{itemize}


\item
{\bf Differentiate Data Types}\\
The following features are classified as:
\begin{itemize}
    \item Height of buildings in Switzerland: Continuous (Numerical)
    \item Eye Color of students at USI:  Nominal (Categorical)
    \item Number of Siblings of Data Science students: Discrete (Numerical)
    \item Favorite Food of University Professors:  Nominal (Categorical)
    \item Pixel brightness value on digital camera: Continuous (Numerical, subject to truncation)
    \item Final ranking of 24 teams in UEFA Euro: Ordinal (Categorical)
    \item Survival time of breast cancer patients: Continuous (Numerical, subject to censoring)
\end{itemize}
\textbf{A note for the students}: We had a brief discussion during the class about why we would want rankings to be
	ordinal instead of discrete, one argument for them being ordinal can be the fact that rankings
	provide information about the relative positions (1st, 2nd, 3rd, etc.), but do not offer a measure
	of how much better or worse one rank is compared to another. In an ordinal ranking, you know the order 
	but not the magnitude of difference between the positions. For example, if Germany ranked 12th in the 
	Euros, Switzerland ranked 11th, and Italy ranked 13th, this tells you their relative order but not how
	much better Switzerland was compared to Germany or Italy. In contrast, a discrete scale would involve
	meaningful, fixed differences between values. However, rankings do not inherently convey such fixed differences,
	which is why they are categorised as ordinal.

\item
{\bf Create Data Records}\\
A data table for three employees is as follows:

\begin{table}[h!]
    \centering
    \begin{tabular}{|c|c|c|c|}
    \hline
    \textbf{Employee ID} & \textbf{Name} & \textbf{Department} & \textbf{Salary (CHF)} \\
    \hline
    101 & John & Data Science & 90,000 \\
    102 & Jane & HR & 80,000 \\
    103 & Max & Marketing & 85,000 \\
    \hline
    \end{tabular}
\end{table}

\item
{\bf Create Data Record from Frequency Table}\\
Based on the frequency table:

\begin{table}[h!]
    \centering
    \begin{tabular}{|c|c|c|c|}
    \hline
    \textbf{ID} & \textbf{Major (Bachelor)} & \textbf{Gender} & \textbf{Status}\\
    \hline
	1 & Data Science & Male & Admitted\\
	2 & Computer Science & Female & Denied\\
	3 & Computer Science & Female & Admitted\\
	4 & Data Science & Female & Denied\\
	5 & Computer Science & Male & Denied\\
	6 & Data Science & Female & Admitted\\
	\vdots & \vdots & \vdots & \vdots\\
	146 & Computer Science & Male & Admitted\\
    \hline
    \end{tabular}
\end{table}

\item
{\bf Create Your Own Data Record}\\
This exercise is up to how you interpret and analyse the picture. One idea is 
that each picture can be one statistical unit. The features can include the number of 
people in a painting, the colour palette of the picture, the aspect ratio, etc.


\end{enumerate}

\end{document}
